\section {Die Fachschaftsvollversammlung}

\subsection{Fachschaftsvollversammlung}
\begin{enumerate}
\item Die Fachschaftsvollversammlung ist das oberste, beschlussfassende Gremium der Fachschaft Informatik.
\item Der Fachschaftsvollversammlung gehören alle Mitglieder der Fachschaft Informatik an.
\item Alle Mitglieder der Fachschaft Informatik haben in der Fachschaftsvollversammlung Antrags-, Rede- und Stimmrecht.
\item Die Fachschaftsvollversammlung ist den Mitgliedern des Fachschaftsrates gegenüber Weisungsberechtigt und nimmt deren Berichte entgegen.
\item Die Fachschaftsvollversammlung gibt sich eine eigene Geschäftsordnung sowie eine Wahlordnung für die Wahlen zum Fachschaftsrat. Die Grundlage dieser Wahlordnung ist die Wahlordnung der Studierendenschaft.
\end{enumerate}

\subsection{Einberufung}
\begin{enumerate}
\item Die Fachschaftsvollversammlung muss einmal in jedem Semester vom Fachschaftsrat einberufen werden. Sie ist ferner einzuberufen:
\begin{itemize}
\item auf Antrag von mindestens fünf Prozent der Fachschaftsmitglieder.
\item auf Antrag der Mehrheit der studentischen Vertreter im Fachbereichsrat.
\end{itemize}
\item Der Fachschaftsrat sorgt für die Einberufung der Fachschaftsvollversammlung. Die Durchführung erfolgt in Mitarbeit der Antragssteller.
\item Die Einberufung der Fachschaftsvollversammlung wird durch den Fachschaftsrat an mehreren, für die Studierenden frei zugänglichen Stellen bekannt gegeben. Die Art der Bekanntmachung soll möglichst alle Fachschaftsmitglieder erreichen. Der Aushang muss die Tagesordnung enthalten und mindestens zwei Wochen innerhalb der Vorlesungszeit vor Beginn der Fachschaftsvollversammlung erfolgen.
\end{enumerate}

\subsection{Beschlussfähigkeit}
\label{sec:beschlussfaehigkeit}
\begin{enumerate}
\item Die Fachschaftsvollversammlung ist beschlussfähig bei Anwesenheit von mindestens zehn Prozent der Fachschaftsmitglieder.
\item \label{auszerordentlich} Bei Anwesenheit von weniger als zehn Prozent der Fachschaftsmitglieder ist eine außerordentliche Vollversammlung innerhalb von vierzehn Tagen, früstens jedoch nach 48 Stunden mit den gleichen Tagesordnungspunkten einzuberufen. Diese Fachschaftsvollversammlung ist dann ohne Rücksicht auf die Teilnehmerzahl beschlussfähig.
\end{enumerate}

\subsection{Beschlussfassung}
\begin{enumerate}
\item Bei einer ordentlichen Fachschaftsvollversammlung werden die Beschlüsse mit der Mehrheit der anwesenden Teilnehmer gefasst. Übersteigt die Anzahl der Enthaltungen die Summe der Für- und Gegenstimmen, so gilt der Antrag als abgelehnt. Bei gleicher Anzahl der Für- und Gegenstimmen (Stimmengleichheit) wird nach nochmaliger Debatte über den Tagesordnungspunkt erneut abgestimmt. Ergibt sich wiederum eine Stimmengleichheit, so gilt der Antrag ebenfalls als abgelehnt.
\item Die außerordentliche Fachschaftsvollversammlung nach \ref{sec:beschlussfaehigkeit} Abs.~\ref{auszerordentlich} ist ohne Rücksicht auf die Teilnehmerzahl beschlussfähig. Beschlüsse können jedoch nur mit Zweidrittelmehrheit der anwesenden Teilnehmer gefasst werden.
\end{enumerate}

\subsection{Protokoll}
\begin{enumerate}
\item Der Sprecher bestimmt bei jeder Vollversammlung einen Protokollführer aus den Mitgliedern der Vollversammlung.
\item Das Protokoll wird vom Sprecher sowie vom Protokollführer unterzeichnet. Ist der Protokollführer ein Vorstandsmitglied des Fachschaftsrates, so muss ein weiteres anwesendes Mitglied der Vollversammlung unterzeichnen.
\item Die Protokolle müssen öffentlich zugänglich bereit gestellt werden, sowie in gedruckter Form im Fachschaftsraum gesammelt und archiviert werden.
\end{enumerate}
