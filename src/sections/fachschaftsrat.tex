\section{Der Fachschaftsrat}
\subsection{Aufgaben und Tätigkeiten}
\begin{enumerate}
\item Der Fachschaftsrat Informatik führt zwischen den Vollversammlungen die Geschäfte der Fachschaft. Vom Fachschaftsrat beauftragte Fachschaftsmitglieder können ebenfalls im Namen der Fachschaft oder des Fachschaftsrates tätig werden.
\item Der Fachschaftsrat Informatik stellt die Vertretung der Studierenden des Fachbereichs im Studierendenparlament sicher.
\item Auf Verlangen des AStA, jedoch mindestens einmal im Monat unaufgefordert, legt der Fachschaftsrat einen Rechenschaftsbericht über die vom AStA zur Verfügung gestellten Gelder ab. Näheres regelt die Finanzordnung.
\item Der Fachschaftsrat weist dem AStA in jeder Legislaturperiode die odrnungsgemäße Wahl seiner Mitglieder nach.
\item Ziel der Arbeit des Fachschaftsrats ist
\begin{itemize}
\item Hilfe für die Fachschaftsmitglieder bei auftretenden Problemen.
\item Förderung des gesellschaftlichen Zusammenhalts der Studenten, Professoren, Assistenten und anderer Mitarbeiter im Fachbereich Informatik.
\item Förderung der Kommunikation und des Erfahrungsaustauschs innerhalb des Fachbereichs, sowie mit anderen Fachbereichen und Fachschaften.
\item durch geeignete Maßnahmen das Studium der Fachschaftsmitglieder angenehmer zu gestalten.
\end{itemize}
\item Der Fachschaftsrat betreut die Sammlung an Informatik-Klausuren der Fachschaft.
\item Desweiteren kann der Fachschaftsrat weitere Dienstleistungen anbieten.
\item Arbeiten und Aufgaben des Fachschaftsrats sollen von allen Mitgliedern in möglichst gleichen Umfang übernommen werden.
\item Fachschaftsmitglieder sollten an der Arbeit des Fachschaftsrats beteiligt werden.
\end{enumerate}