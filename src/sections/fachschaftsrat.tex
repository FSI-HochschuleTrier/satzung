\section{Der Fachschaftsrat}
\subsection{Aufgaben und Tätigkeiten}
\begin{enumerate}
\item Der Fachschaftsrat Informatik führt zwischen den Vollversammlungen die Geschäfte der Fachschaft. Vom Fachschaftsrat beauftragte Fachschaftsmitglieder können ebenfalls im Namen der Fachschaft oder des Fachschaftsrates tätig werden.
\item Der Fachschaftsrat Informatik stellt die Vertretung der Studierenden des Fachbereichs im Studierendenparlament sicher.
\item Auf Verlangen des AStA, jedoch mindestens einmal im Monat unaufgefordert, legt der Fachschaftsrat einen Rechenschaftsbericht über die vom AStA zur Verfügung gestellten Gelder ab. Näheres regelt die Finanzordnung.
\item Der Fachschaftsrat weist dem AStA in jeder Legislaturperiode die odrnungsgemäße Wahl seiner Mitglieder nach.
\item Ziel der Arbeit des Fachschaftsrats ist
\begin{itemize}
\item Hilfe für die Fachschaftsmitglieder bei auftretenden Problemen.
\item Förderung des gesellschaftlichen Zusammenhalts der Studenten, Professoren, Assistenten und anderer Mitarbeiter im Fachbereich Informatik.
\item Förderung der Kommunikation und des Erfahrungsaustauschs innerhalb des Fachbereichs, sowie mit anderen Fachbereichen und Fachschaften.
\item durch geeignete Maßnahmen das Studium der Fachschaftsmitglieder angenehmer zu gestalten.
\end{itemize}
\item Der Fachschaftsrat betreut die Sammlung an Informatik-Klausuren der Fachschaft.
\item Desweiteren kann der Fachschaftsrat weitere Dienstleistungen anbieten.
\item Arbeiten und Aufgaben des Fachschaftsrats sollen von allen Mitgliedern in möglichst gleichen Umfang übernommen werden.
\item Fachschaftsmitglieder sollten an der Arbeit des Fachschaftsrats beteiligt werden.
\end{enumerate}

\subsection{Wahl und Zusammensetzung}
\label{sec:zusammensetzung}
\begin{enumerate}
\item Der Fachschaftsrat setzt sich aus 15, mindestens jedoch fünf Mitgliedern zusammen. Die Mitgliedschaft in AStA, StuPa und Fachschaftsrat schließen sich gegensietig aus. Die Wahl erfolgt in einfacher Mehrheitswahl nach Maßgabe der Wahlordnung der Studierendenschaft der Hochschule Trier.
\item Wahlberechtigt und Wählbar ist jedes Mitglied der Fachschaft Informatik der Hochschule Trier.
\item Eine Briefwahl ist gemessen an der Unverhätnismäßgkeit des Aufands nicht möglich.
\item Der Wahlausschuss besteht aus mindestens vier Mitgliedern der Fachschaft Informatik.
\item Für die Wahl des Fachschaftsrates gilt folgendes Verfahren:
\begin{itemize}
\item Die Wahl wird in einfacher Mehrheitswahl durchgeführt.
\item Die Wahlberechtigten erhalten einen Wahlzettel, auf dem die Kandidaten benannt sind.
\item Die Wählenden haben höchstens fünf Stimmen.
\item Bei Stimmgleichheit entscheidet das Los.
\item Kandidaten mit mindestens einer Stimme sind automatisch Ersatzmitglieder des Fachschaftsrates, wenn sie ihre Wahl innerhalb der dazu bestimmten Frist annehmen.
\end{itemize}
\item \label{ersatz} Gibt es keine Ersatzmitglieder und ist der Fachschaftsrat nicht voll besetzt, so können durch einstimmigen Beschluss weitere Mitglieder aus der Fachschaft aufgenommen werden. Diese Mitglieder haben die gleichen Rechte und Pflichten wie die von der Fachschaft gewählten Mitglieder.
\item Der Wahlprüfungsausschuss besteht aus vier Mitgliedern, die nicht zur Wahl stehen sollen, von denen mindestens eins dem amtierenden Fachschaftsrat angehört.
\end{enumerate}

\subsection{Amtszeit}
\begin{enumerate}
\item Die Amtszeit des Fachschaftsrats beträgt ein Jahr und beginnt mit der konstituierenden Sitzung.
\item Scheidet ein Mitglied des Fachschaftsrats vorzeitig aus, so rückt die nächste Kandidatin  oder der nächste Kandidat nach. Steht kein Kandidat zur Verfügung kann der Fachschaftsrat nach \ref{sec:zusammensetzung} Abs.~\ref{ersatz} den Platz besetzten.
\item Die Amtszeit eines Ratsmitglieds endet vorzeitig:
\begin{itemize}
\item durch Exmatrikulation
\item durch Verzicht, welcher dem Vorstand des Fachschaftsrats schriftlich mizuteilen ist.
\end{itemize}
\item Der Fachschaftsrat kann aufgelöst werden:
\begin{itemize}
\item auf Beschluss seiner Mitglieder mit Zweidrittelmehrheit
\item durch Beschluss der Fachschaftsvollversammlung
\item sofern nur noch weniger als fünf Mitglieder im Fachschaftsrat sind
\end{itemize}
\item Findet die Neuwahl des Fachschaftsrats in der zweiten Hälfte der Legislaturperiode des aufgelösten Fachschaftsrats statt, so verlängert sich die Amtszeit des neu gewählten Fachschaftsrats automatisch bis zum übernächsten regulären Wahltermin. Bei einer Neuwahl in der ersten Hälfte der Legislaturperiode verkürzt sich die Amtszeit auf entsprechend auf den regulären Wahltermin.
\item Mit Beendigung der Amtszeit ist das Mitglied verpflichtet in seinem Besitz befindliches Eigentum oder Vermögen der Fachschaft unverzüglich zurückzugeben.
\item Vom scheidenden Mitglied ausgeübte Amäter werden mit Ende der Amstzeit abgelegt.
\end{enumerate}