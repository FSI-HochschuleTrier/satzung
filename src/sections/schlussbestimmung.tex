\section{Schlussbestimmung}
\subsection{Nicht-Existenz des Fachschaftsrates}
Bei nicht existierendem Fachschaftsrat haben die studentischen Mitglieder des Fachbereichsrates unverzüglich für die Durchführung von Neuwahlen zu sorgen.

\subsection{Satzungsänderungen}
\begin{enumerate}
\item Satzungsänderungen dürfen nur im Rahmen der Satzung der Studierendenschaft der Hochschule Trier und der geltenden Gesetze erfolgen. Nicht inhaltliche Änderungen können ohne Beschluss der Fachschaftsvollversammlung vom Fachschaftsrat durchgeführt werden.
\item Geplante Satzungsänderungen sollen mindestens zwei Wochen vor der beschließenden Fachschaftsvollversammlung zur Einsicht zugänglich gemacht werden.
\item Auf Nachfrage eines Fachschaftsmitgliedes müssen die Änderungen im Vergleich zur alten Satzung erläutert werden.
\item Dem StuPa und dem AStA sind je ein Exemplar der neuen Satzung zur Verfügung zu stellen.
\end{enumerate}

\subsection{Geschäftsordnung und Wahlordnung}
Diese Satzung beinhaltet die Geschäftsordnung der Fachschaftsvollversammlung, sowie die Wahlordnung für die Wahlen zum Fachschaftsrat.

\subsection{Inkrafttreten}
Diese Satzung tritt nach Beschlussfassung durch die Fachschaftsvollversammlung in Kraft. Gleichzeitig tritt die alte Satzung der Fachschaft Informatik der Hochschule Trier, beschlossen am 31.10.2013, außer Kraft.
