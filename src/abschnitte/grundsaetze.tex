\section{Grundsätze}

\subsection{Allgemeines}
\begin{enumerate}
	\item Die Studenten des Fachbereichs Informatik der Hochschule Trier bilden die Fachschaft Informatik.
	\item Die Fachschaft Informatik ist Teil der Studierendenschaft der Hochschule Trier und verwaltet ihre Angelegenheiten selbst.
	\item Studenten des Fachbereichs Informatik im Sinne dieser Satzung ist jeder in einem Studiengang des Fachbereichs Informatik immatrikulierte Student der Hochschule Trier.
\end{enumerate}

\subsection{Rechte und Pflichten der Fachschaftsmitglieder}
\label{par:rechte-und-pflichten}

\begin{enumerate}
	\item Jedes Fachschaftsmitglied hat das Recht in den Organen der Fachschaft mitzuwirken.
	\item Jedes Fachschaftsmitglied hat nach \S 2 Abs. 2 der Satzung der Studierendenschaft der Hochschule Trier das aktive und passive Wahlrecht.
	\item Jedes Fachschaftsmitglied hat das Recht in den Organen der Fachschaft gehört zu werden und Anträge zur Beschlussfassung vorzulegen.
\end{enumerate}

\subsection{Organe der Fachschaft}
Die Fachschaft Informatik bildet folgende Organe:

\begin{itemize}
	\item die Fachschaftsvollversammlung
	\item den Fachschaftsrat
\end{itemize}
